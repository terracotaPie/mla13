\documentclass{article}
\usepackage{mla13}
\title{Institutional Histories}
\firstname{Timur}
\lastname{Borkhodoev}
\professor{Professor John Semley}
\class{CIN210H Horror Movies}
\sources{SampleBibDocument.bib}
\begin{document}
\makeheader

Hammer Films or Hammer Pictures is a British film
production founded by comedian and a businessman William
Hinds in 1934. The company is well known for a series of
Gothic films made in period from the mid-1950's, until
the 1970's. We are interested in the story of overnight
success, critics reception - and company's ``Big Boss''
at that time sir James Carreras.  

Horror genre was basically dead since the end of the war.
The new British ``X'' certificate, was introduced being
the first compulsary - ``Suitable for those aged 16 and
older'' - was not helpful. However genre of science
fiction emerged with the docudrama ``Destination Moon''.
Later we see, Frankenstein’s monster soon appeared in the guise of Martians in Invaders
from Mars (1953), while the spot of Dracula had
had been taken by the blood-sucking creature of
The Thing from Another World (1951).  

Fortunately for Hammer Films in 1947 the government
imposed 75\% import tax on American Movies - ultimately
halting the flow of movies to Britian. It creates demand -
cinemas soon become starved for product. James Carreras
considers such turn of events a great opportunity - and
in 1948 ``Dick Barton: Special Agent'' was released and
immediately became a huge success. Ban was lifted before
the summer of 1948 - though audience was already hooked on
Dick Barton and demanded more. Carreras delivers
and followes the simple practice - ``More of the same''.
First movie is quickly followed by another two episodes: 
``Dick Barton Strikes Back''(1949) and ``Dick Barton at
Bay''(1950).  

We start seeing transition to darker narrative in
``Dr.Morelle: The case of Missing Heiress''. However
first Appearance of grotesque villain wasn't very
succesful.  

Carrera's in attempt to attract audience once again
announces his new technical crew headed by director
Godfrey Grayson and screenwriter Rawlison(well known
at that time for Hitchcock's ``The man who knew too
much''). The price paid for the names is high - which
forces Hammer Productions to move from Dial Close(1949)
to house owned by Ernest Olivier at the price of mere
50 Guineas(British gold coin). He intensifies the hype
not only with famous names, but with schedule of 8 movies
by the end of the year. Having intense layout:
4 Weeks on - 2 week off. Marketing schemes used by sir
James Carreras proved to be succesfull.

Huge leap was made in just over 2 years to large-scale
distributor. 13 branches, over a hundred employees and a
luxury office on the fifth floor of 113 Wardour Street
(nowadays knows simply as Hammer House).

\makeworkscited
\end{document}
